\documentclass[Bachelorarbeit.tex]{subfiles}
\begin{document}
% how to prove that the final implementation works
\chapter{Test scenario}
\section{Target}
The target of the implementation is to use the video from a web cam to test the implemented color based face detection. The implemented code should as simple as possible, to achieve a real-time capable 

\section{Implementation steps}
Following steps will be done to test if the implemented solution is real-time capable.

\subsection{Color based face detection on a picture}
The first step is to test the algorithms on different pictures. For this following steps are scheduled:
\begin{enumerate}
\item Transform picture into the YCbCr color space.
\item Find suitable threshold ranges for the YCbCr.
\item Make Thresholding on the YCbCr to get a binary picture.
\item Detecting faces out of skin regions.
\item Draw boxes to identify the faces on the picture.
\end{enumerate}
\subsection{Color based face detection on a video}\label{CbVidoe}
Use a web cam and test the implemented color based algorithm.
\subsection{Real-time Color based face detection on a picture}
There are three possibilities to test if the implemented solution is real-time capable.
\begin{enumerate}
\item Test it on a notebook and a web cam (same as section \ref{CbVidoe}).
\item Calculate if the computation time is less enough for a micro controller like a Arduino.
\item Implement the algorithm on a micro controller which is connected with a web cam.
\end{enumerate}
The decision which method will be implemented/tested depends on the time reserve after the points before were implemented.
\end{document}